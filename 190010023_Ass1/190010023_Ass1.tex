\documentclass{article}
\usepackage[utf8]{inputenc}
\usepackage{xcolor}
\usepackage{blindtext}
\renewcommand{\baselinestretch}{0.9}
\usepackage{url}
\usepackage{graphicx}
\usepackage{amsmath}
\usepackage{mathtools}
\graphicspath{{./images/}}
\usepackage{algorithm}% http://ctan.org/pkg/algorithms
\usepackage{algcompatible}% http://ctan.org/pkg/algorithmicx

\begin{document}


\title{\textbf{Assignment 1} }
\author{Manav Kushwaha \\
2nd Year, CSE, IIT Dharwad \\
\texttt{190010023@iitdh.ac.in}}
\date{\today}
\maketitle

\begin{figure}[h]
    \centering
    \includegraphics{images/IITDH.jpg}
    \caption{IIT-Dharwad}
    \label{fig:IIT-Logo}
\end{figure}

\newpage
\tableofcontents
\listoffigures
\listoftables

\newpage

\section{Maths Section}
\label{sec:math}
    \hfill
    \subsection{Inline mathematical expression}
         $$C(n,r) = n!/(r! (n - r)!)$$
    
    \subsection{Non-numbered Equation}
        $$a = b + c/d$$
        
    \subsection{Numbered Equation}
        $$10*2+5=25$$
        
    \subsection{Multiline Equation}
        \begin{equation}    \label{eq:multiline eq}
        \begin{aligned}
            h = & \sqrt{(a+b)^2-4ab} \\ 
              = & \sqrt{(a-b)^2} \\  
              = & |a-b|   
        \end{aligned}
        \end{equation}
    
    \subsection{Matrices}
        \begin{equation}
            \centering  
            \begin{bmatrix}
            a & b & c \\
            1 & 2 & 3\\
            
            \end{bmatrix}
            \begin{bmatrix}
            1 & 0 & 0 \\
            0 & 1 & 0\\
            0 & 0 & 1\\
            \end{bmatrix}
            =
            \begin{bmatrix}
                a & b & c \\
                1 & 2 & 3\\
            \end{bmatrix}
        \end{equation}
    
    \subsection{Square root}
        $$\sqrt{a+b*c}$$
    
    \subsection{Summation}
        $$\sum_{i=0}^{10}i = 10*(10+1)/2 = 55$$
        
    \subsection{Integration}
        $$\int_0^{\pi/2}sinx~dx = 1$$
    
    \subsection{Nested Brackets}
        \[
           \Bigg\{xy\bigg\{z \Big\{\frac{x_1}{y_1}\Big\}\bigg\}\Bigg\}
        \]
    \subsection{Fractions}
        \[
            \Bigg\{\frac{\bigg(\frac{a}{b}\bigg)}{c}\Bigg\}
        \]
\section{Font Styles}
        The font settings are following:
        
        \hfill
        
        \textbf{Bold-Font}
        
        \textit{Italic}
        
        \texttt{This is an example of teletype font} 
        
        \textsc{This is an Example of Small Capitals}
\section{Colors}
    The color settings are following:
    
    \textcolor{orange}{This text is orange}
    
    \colorbox{cyan}{The colour of the text background is Cyan}
    
    \pagecolor{yellow}{The colour of the background page is yellow}
    \newpage
    \pagecolor{white}

\section{Lists}

\textbf{This is an example of mixed lists}

\begin{enumerate}
    \item You can mix list environments as much as you like
    \begin{itemize}
        \item But it might start to look silly
        \item[*] With different symbols
    \end{itemize}
    \item So do remember
    \begin{description}
        \item[Word 1] This is the definition of the word 1.
        \item[Word 2] This is the definition of the word 2.
    \end{description}
\end{enumerate}

\section{Referencing and Crosslinking}

The examples are the following:
\hfill

We have seen various examples in the Maths section i.e. section~\ref{sec:math}

We have seen the Logo of IIT Dharwad, Figure~\ref{fig:IIT-Logo} on the page~\pageref{fig:IIT-Logo}

We have the multilined equation~\ref{eq:multiline eq} on Page~\pageref{eq:multiline eq}

% \newpage
% % This is a comment line :)
\section{Tables}

We have a multicolumn and multirow table as an example in this section.


\begin{table}[h]
    \centering
    \begin{tabular}{||p{2cm} |p{2cm} |p{2cm} |p{2cm} ||}
    \hline
     1,1 & 1,2 & 1,3 & 1,4  \\
     \hline
     \hline
     2,1 & 2,2 & 2,3 & 2,4 \\
     \hline
     3,1 & 3,2 & 3,3 & 3,4 \\
     \hline
     4,1 & 4,2 & 4,3 & 4,4 \\
     \hline 
\end{tabular}
    \caption{Test-Table}
    \label{tab:Test-Table}
\end{table}

\newpage

\section{Pseudocode of Quicksort}
% \textbf{This is the pseudocode of quicksort}

\begin{algorithm}
        \caption{Quicksort}
        \label{algorithm:quicksort}
        \begin{algorithmic}[1]
            \STATE Given: Array a and it's size n.
            \STATE
            \STATE We start by passing array into the function along with end positions of the array
            \STATE 
            \STATE Quicksort(Array a,int p,int r) \hfill    ....\COMMENT {p and r are initial and final positions}
                \begin{algorithmic}[1]
                    \IF{$i < f$}
                    \STATE $q = Partition(a,p,r)$
                    \STATE $Quicksort(a,p,q)$
                    \STATE $Quicksort(a,q+1,r)$
                    \ENDIF  
                \end{algorithmic}
            \STATE
            \STATE Partition(Array a,int p,int r)
            \begin{algorithmic}[1]
                \STATE $x = a[r]$ \hfill     .....\Comment{Choosing pivot}
                \STATE $i = p-1$
                \STATE $j = r+1$
                \STATE
                \FOR{$j = p$ \textbf{to} $r-1$}
                \IF{$a[j] \leq x$}
                \STATE $i = i+1$
                \STATE exchange $a[i]$ with $a[j]$
                \ENDIF
                \ENDFOR
                
                \noindent \STATE exchange $a[i+1]$ with $a[r]$
                \STATE \textbf{return} $i+1$
               
            \end{algorithmic}
            
        \end{algorithmic}
\end{algorithm}
\newpage
\section{Bibliography}
\bibliographystyle{ieeetr}
\bibliography{references}

In \cite{Cormen2009IntroductionTA}, The authors have explained the quicksort pseudocode. This pseudocode was used in Algorithm~\ref{algorithm:quicksort}.

In \cite{qsort}, The author has given the pseudocode for the quicksort which inturn helped me to gain insight into the algorithm

In \cite{9117073}, The Authors has researched on dectecting Regions At Risk for Spreading COVID-19 Using Existing Cellular Wireless Network Functionalities

In \cite{1527552}, The authors have explained about machine learning.

In \cite{5641441}, The author have explained about artificial intelligence.

\end{document}